% Created 2018-04-19 周四 12:41
% Intended LaTeX compiler: pdflatex
\documentclass[11pt]{article}
\usepackage[utf8]{inputenc}
\usepackage[T1]{fontenc}
\usepackage{graphicx}
\usepackage{grffile}
\usepackage{longtable}
\usepackage{wrapfig}
\usepackage{rotating}
\usepackage[normalem]{ulem}
\usepackage{amsmath}
\usepackage{textcomp}
\usepackage{amssymb}
\usepackage{capt-of}
\usepackage{hyperref}
\date{\today}
\title{}
\hypersetup{
 pdfauthor={},
 pdftitle={},
 pdfkeywords={},
 pdfsubject={},
 pdfcreator={Emacs 25.3.1 (Org mode 9.1.9)}, 
 pdflang={English}}
\begin{document}

\tableofcontents

\section{Emaces Usful Command}
\label{sec:orgdddd02a}
\subsection{Command}
\label{sec:org58a34ee}
\begin{verbatim}
cmd --insecure 模式打开emacs才能有网络连接
cmd --insecure 用Texlive GUI command-line打开emacs之后默认用texlive编译.tex
c-x c-s 保存当前文件
m-x list-package 列出所有package
c-x o 切换窗口
m-> 文档底部
m-< 文档开头
c-v 上滚屏
m-v 下滚屏
c-x c-b 列出所有缓冲区/没什么用 直接用打开文件就行了
c-x h 全选
c-space 标记
c-@     标记
c-/ 撤销命令
c-_ 撤销命令
M-m 打开spacemacs主菜单
M-m TAB一键循环切换buffer
c-e M-b 先切换到行尾 然后按句子单位回退
M-e M-a 换行 到指定位置 最好用的命令
\end{verbatim}
\subsection{帮助}
\label{sec:org0d824e6}
\begin{verbatim}
c-h 主命令
c-h a 查找关键词对应的函数
c-h f 函数
c-h v 变量
c-h k 快捷键
\end{verbatim}
\section{Latex 编辑命令}
\label{sec:orgf2a7e1f}
\subsection{中文环境包}
\label{sec:orgd34735a}
\begin{verbatim}
% 中文支持包
\usepackage{ctex}
\usepackage{CJK}

 % 调用环境变量 
 \begin{CJK}{UTF8}{song}
 some content here.
 具体内容
 \end{CJK}
\end{verbatim}
\section{org mode}
\label{sec:orgd1d7b30}
\subsection{Useful Command}
\label{sec:org1a8ba1a}
\begin{verbatim}
TAB 切换标题
s-TAB 切换标题
m-left/right 升降级标题
m-enter 插入一个同级标题
<s TAB 快速插入一个 源代码块标签
<e TAB 快速插入一个 example块标签
s    #+begin_src ... #+end_src   
e    #+begin_example ... #+end_example  : 单行的例子以冒号开头  
q    #+begin_quote ... #+end_quote      通常用于引用,与默认格式相比左右都会留出缩进  
v    #+begin_verse ... #+end_verse      默认内容不换行,需要留出空行才能换行  
c    #+begin_center ... #+end_center   
l    #+begin_latex ... #+end_latex   
L    #+latex:   
h    #+begin_html ... #+end_html   
H    #+html:   
a    #+begin_ascii ... #+end_ascii   
A    #+ascii:   
i    #+index: line   
I    #+include: line 
\end{verbatim}
\begin{itemize}
\item 单纯文字编辑 空一行即可
\item 可以用- 来表示要点提示的项目符号
\item 可以用上述块标签来表示源代码等
\end{itemize}
\end{document}
