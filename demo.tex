% Created 2018-04-22 周日 23:40
% Intended LaTeX compiler: pdflatex
\documentclass[11pt]{article}
\usepackage[utf8]{inputenc}
\usepackage[T1]{fontenc}
\usepackage{graphicx}
\usepackage{grffile}
\usepackage{longtable}
\usepackage{wrapfig}
\usepackage{rotating}
\usepackage[normalem]{ulem}
\usepackage{amsmath}
\usepackage{textcomp}
\usepackage{amssymb}
\usepackage{capt-of}
\usepackage{hyperref}
\date{\today}
\title{}
\hypersetup{
 pdfauthor={},
 pdftitle={},
 pdfkeywords={},
 pdfsubject={},
 pdfcreator={Emacs 25.3.1 (Org mode 9.1.9)}, 
 pdflang={English}}
\begin{document}

\tableofcontents

\section{Emaces Usful Command}
\label{sec:orgf9dc366}
\subsection{Command}
\label{sec:org66c83a4}
\subsubsection{常用命令}
\label{sec:org7310022}
\begin{verbatim}
cmd --insecure 模式打开emacs才能有网络连接
cmd --insecure 用Texlive GUI command-line打开emacs之后默认用texlive编译.tex
c-x c-s 保存当前文件
m-x list-package 列出所有package
c-x o 切换窗口
m-> 文档底部
m-< 文档开头
c-v 上滚屏
m-v 下滚屏
c-x c-b 列出所有缓冲区/没什么用 直接用打开文件就行了
c-x h 全选
c-space 标记
c-@     标记
c-/ 撤销命令
c-_ 撤销命令
M-m 打开spacemacs主菜单
M-m TAB一键循环切换buffer
c-e M-b 先切换到行尾 然后按句子单位回退
M-e M-a 换行 到指定位置 最好用的命令
M-; 注释掉当前行
M-m 0-9 提供于windows-numbering包 作用:当有多个buffer同时打开时,一键切换当前选中buffer
\end{verbatim}
\subsubsection{命令主菜单}
\label{sec:org46e8a50}
\begin{verbatim}
M-m spacemacs buffer管理
M-x helm-command
c-c 当前文档格式 编译菜单
c-x 文档保存 新建等管理
c-h 帮助
\end{verbatim}
\subsubsection{帮助}
\label{sec:org2d8f419}
\begin{verbatim}
c-h 主命令
c-h a 查找关键词对应的函数
c-h f 函数
c-h v 变量
c-h k 快捷键
\end{verbatim}
\subsection{一些解决方案}
\label{sec:org6497394}
\subsubsection{多buffer管理:}
\label{sec:org41d4c6e}
先分屏 后打开buffer
\subsubsection{快捷跳转行数}
\label{sec:orgd5bfdfe}
\begin{verbatim}
M-g g 跳转到第几行  需要设置显示行数功能
\end{verbatim}
\subsubsection{设置在左侧显示行数功能}
\label{sec:org92d0b6b}
\begin{enumerate}
\item 绑定按键及配置文件
\label{sec:org767cee0}
\begin{verbatim}
spacemacs build in function
C-x t 绑定按键到M-x linum-mode
\end{verbatim}
绑定按键的配置文件 修改init.el文件
\begin{verbatim}
;; global-set-key
;;
;; set linum-mode
(global-set-key (kbd "C-x t") 'linum-mode)
\end{verbatim}
\item 特别注意需要linum-relative package的支持
\label{sec:orgf0219d5}
在.spaces中初始化才能startup载入
\item 可以直接搜索变量 C-h v 然后配置变量
\label{sec:org67cefcc}
好处是,可以通过UI界面来配置初始化变量
\end{enumerate}
\subsubsection{复制粘贴及矩形区域选择解决方案}
\label{sec:orgdee75c2}
\begin{enumerate}
\item 利用expand region的快捷键
\label{sec:org751d06d}
\begin{verbatim}
M-m v 标记并可以调整区域
\end{verbatim}
\item 利用C-x C-x 瞬间交换光标的位置与开头或者结尾
\label{sec:orgbeddc1d}
完美!解决!
\end{enumerate}
\subsection{Spacemacs}
\label{sec:org308782f}
\subsubsection{Spacemacs 配置文件 添加package cuda-mode}
\label{sec:orgad9b340}
添加该段代码到.spacemacs.el文件中的合适位置
\begin{verbatim}
;; add packages 
;;
(defun dotspacemacs/layers ()
  "Configuration Layers declaration."
  (setq-default
   ;; ...
   ;; List of additional packages that will be installed wihout being
   ;; wrapped in a layer. If you need some configuration for these
   ;; packages then consider to create a layer, you can also put the
   ;; configuration in `dotspacemacs/config'.
   ;; add packages
   dotspacemacs-additional-packages '(
                                      cuda-mode
                                      company
                                      )
   ;; ...
   ))

(defun dotspacemacs/config ()
  "Configuration function.
This function is called at the very end of Spacemacs initialization after
layers configuration."
  ;; add packages
  dotspacemacs-additional-packages '(
                                     cuda-mode
                                     company
                                     )
  ;; 
  )
\end{verbatim}
\subsubsection{Spacemacs 配置文件 初始化init.el文件 使所有buffer默认打开company模式}
\label{sec:orgdff41c3}
\begin{verbatim}
;; Enable global company mode
(require 'company)
(add-hook 'after-init-hook 'global-company-mode)
(setq company-idle-delay 0.1)
(setq company-minimum-prefix-length 1)
(setq company-backends '((company-capf company-files company-elisp company-inf-ruby company-anaconda company-go company-irony company-clang company-cmake company-css company-yasnippet) (company-dabbrev company-dabbrev-code)))
\end{verbatim}
\subsection{Company 自动补全包 命令集}
\label{sec:orgc1c8e24}
\begin{verbatim}
M-n M-p select
Enter: to complete 
C-s, C-r and C-o: Search through the completions with 
M-(digit) to quickly complete with one of the first 10 candidates.
\end{verbatim}
\subsection{Expand-region 快捷键文本选中 package}
\label{sec:org139cec7}
\begin{verbatim}
M-m v 选中当前光标所在的单词,继续按v则扩大选区 V则缩小选区 具体命令见下方说明
\end{verbatim}
\subsection{Magit package自动上传本地文档到github}
\label{sec:org7f9c040}
\begin{verbatim}
C-x g 已通过global-set-key自定义绑定 键位 到命令magit-status
\end{verbatim}
在新打开的magit窗口中(通过magit-status命令)
\begin{verbatim}
s 小写s表示git add命令
c 表示commit命令
但以上两个步骤已经通过git-auto-commit自动commit过了
P u 大写P表示push命令
\end{verbatim}
\subsection{Git-auto-commit 每次保存文件自动commit到github 需要在.spaces中初始化才能startup载入}
\label{sec:orgfecd71a}
\subsubsection{配置步骤}
\label{sec:org7a68def}
需要两步:
\begin{enumerate}
\item unsigned 需要在.spaces中初始化才能startup载入 add package
\item gac-automatically-push-p 当变量不为0时,还可以自动push!!!! 在package里面customize这个变量即可
\item 绑定按键 C-x p 自动commit+push
\item 下载下来package之后, 需要运行 M-m ! 打开shell窗口配置github
\item 显示信息: minor mode窗口会有gac标志
\end{enumerate}
\begin{verbatim}
git config --global user.email yyang181@github.com
\end{verbatim}
\subsubsection{使用方法}
\label{sec:org796a858}
当且仅当 处理单个文件的编译时,想要多次一键测试结果 可以开启 gac-mode,也即git-auto-commit-mode

快捷键
\begin{verbatim}
C-x p 打开gac模式,使得保存文件之后自动commit push
\end{verbatim}
\subsubsection{按键配置代码 init.el文件}
\label{sec:orgb332327}
\begin{verbatim}
;; global-set-key
(global-set-key (kbd "C-x p") 'git-auto-commit-mode)
\end{verbatim}
\subsection{Evil-nerd-commenter Package}
\label{sec:orgd578406}
\subsubsection{配置步骤}
\label{sec:org244e717}
unsigned 需要在.spaces中初始化才能startup载入

需要在init.el中配置默认按键
\begin{verbatim}
;; set up default hotkeys for evilnc
;;
;; evil-nerd-commenter
(evilnc-default-hotkeys)
\end{verbatim}
设置按键
\begin{verbatim}
(evilnc-default-hotkeys) 使用默认按键
\end{verbatim}
\subsubsection{使用方法}
\label{sec:org103e932}
\begin{verbatim}
C-u number M-; 注释从当前行开始的 number 行
\end{verbatim}
\subsection{Flycheck}
\label{sec:orga45ff81}
\subsubsection{配置方法}
\label{sec:org36bb828}
package unsigned 需要在.spacemacs中配置

To enable Flycheck add the following to your init file:
\begin{verbatim}
(add-hook 'after-init-hook #'global-flycheck-mode)
\end{verbatim}
需要设置变量的值来激活
\begin{verbatim}
C-h v type flycheck-check-syntax-automatically
把这个变量的值修改即可
\end{verbatim}
\subsection{Git 综述}
\label{sec:org677a0e1}
结合两个package完美一键push到github
\begin{itemize}
\item git-auto-commit: 保存当前文件时自动commit
\item magit: s打开magit status界面
\item magit: P u 一键push到github
\item 前提条件是配置了git config --global
\item 大量文件跟更改可以直接用git desktop
\item 单文件修改调试可以用此文中的快捷键方法
\end{itemize}
\subsection{神器:global-set-key自定义绑定 键位 到命令}
\label{sec:org1a78588}
\subsubsection{方法一 修改init.el文件}
\label{sec:orge2519d1}
\begin{verbatim}
;; global-set-key
(global-set-key (kbd "C-x g") 'magit-status)
\end{verbatim}
\subsubsection{方法二 可能会出现单次设置单次使用}
\label{sec:org6be1651}
\begin{verbatim}
M-x global-set-key 
type 需要绑定的键位 并按enter确认
type 需要绑定的命令 并按enter确认
\end{verbatim}
\section{Latex 编辑命令}
\label{sec:orgaee9715}
\subsection{中文环境包}
\label{sec:orgfd67b16}
\begin{verbatim}
% 中文支持包
\usepackage{ctex}
\usepackage{CJK}

 % 调用环境变量 
 \begin{CJK}{UTF8}{song}
 some content here.
 具体内容
 \end{CJK}
\end{verbatim}
\section{Org mode}
\label{sec:org8a3b66f}
\subsection{Useful Command}
\label{sec:org4bca0dc}
\subsubsection{编译}
\label{sec:orga1a4d2d}
\begin{verbatim}
c-c c-e 编译生成html网站格式
c-c c-e 可选生成latex pdf
\end{verbatim}
\subsubsection{标题}
\label{sec:org3a59570}
\begin{verbatim}
TAB 切换标题
s-TAB 切换标题
m-left/right 升降级标题
m-enter 插入一个同级标题
\end{verbatim}
\subsubsection{块标签}
\label{sec:org0e10d9e}
\begin{verbatim}
<s TAB 快速插入一个 源代码块标签
<e TAB 快速插入一个 example块标签
s    #+begin_src ... #+end_src   
e    #+begin_example ... #+end_example  : 单行的例子以冒号开头  
q    #+begin_quote ... #+end_quote      通常用于引用,与默认格式相比左右都会留出缩进  
v    #+begin_verse ... #+end_verse      默认内容不换行,需要留出空行才能换行  
c    #+begin_center ... #+end_center   
l    #+begin_latex ... #+end_latex   
L    #+latex:   
h    #+begin_html ... #+end_html   
H    #+html:   
a    #+begin_ascii ... #+end_ascii   
A    #+ascii:   
i    #+index: line   
I    #+include: line 
\end{verbatim}
\subsubsection{排版段落格式}
\label{sec:orgfb991fa}
\begin{itemize}
\item 单纯文字编辑 空一行即可
\item 可以用- 来表示要点提示的项目符号
\item 可以用上述块标签来表示源代码等
\textbf{*} Org-page 创建个人主页
\end{itemize}
\subsection{Org mode配置latex环境及常用宏包}
\label{sec:org0d0f622}
中文宏包配置
\begin{verbatim}
#+LATEX_HEADER: \usepackage[colorlinks=true,linkcolor=red]{hyperref}
\end{verbatim}
其它可选命令小结
#+LATEX_HEADER: \usepackage[colorlinks=true,linkcolor=red]{hyperref}
#+LATEX_CLASS: org-article
#+TITLE: Org to \LaTeX
\subsection{Org-page package创建个人主页}
\label{sec:orge0b357c}
From \url{https://github.com/kelvinh/kelvinh.github.com}
\subsubsection{.emacs 文件源代码 手动添加package 注意:目前好像不能用}
\label{sec:org1a21e00}
\begin{verbatim}
;;; the following is only needed if you install org-page manually
(add-to-list 'load-path "path/to/org-page")
(require 'org-page)
(setq op/repository-directory "path/to/your/org/repository")
(setq op/site-domain "http://your.personal.site.com/")
;;; for commenting, you can choose either disqus, duoshuo or hashover
(setq op/personal-disqus-shortname "your_disqus_shortname")
(setq op/personal-duoshuo-shortname "your_duoshuo_shortname")
(setq op/hashover-comments t)
;;; the configuration below are optional
(setq op/personal-google-analytics-id "your_google_analytics_id")
\end{verbatim}
\section{Jupyter notebook}
\label{sec:org54dc93b}
\subsection{常用命令}
\label{sec:org20196fa}
\begin{verbatim}
执行当前cell,并自动跳到下一个cell:Shift Enter
执行当前cell,执行后不自动调转到下一个cell:Ctrl-Enter
是当前的cell进入编辑模式:Enter
退出当前cell的编辑模式:Esc
删除当前的cell:双D
为当前的cell加入line number:单L
将当前的cell转化为具有一级标题的maskdown:单1
将当前的cell转化为具有二级标题的maskdown:单2
将当前的cell转化为具有三级标题的maskdown:单3
为一行或者多行添加/取消注释:Crtl /
撤销对某个cell的删除:z
浏览器的各个Tab之间切换:Crtl PgUp和Crtl PgDn
快速跳转到首个cell:Crtl Home
快速跳转到最后一个cell:Crtl End
\end{verbatim}
\subsection{文件导入}
\label{sec:org4580a48}
\subsubsection{如何将本地的.py文件load到jupyter的一个cell里面}
\label{sec:org04c4e5e}
\begin{verbatim}
%load test.py #test.py是当前路径下的一个python文件
\end{verbatim}
\subsubsection{如何将网络中的.py文件load到jupyter的一个cell里面}
\label{sec:orgf932565}
\begin{verbatim}
 在cell中输入%load http://.....,然后运行该cell,就会将load后面所对应地址的代码load到当前的cell中;
\end{verbatim}
\subsubsection{利用cell运行.py文件}
\label{sec:org13f1b55}
\begin{verbatim}
%run file.py
\end{verbatim}
\section{Python 语言}
\label{sec:org36eb3a6}
\subsection{帮助命令}
\label{sec:org055ef64}
\begin{verbatim}
help() 查询括号里面的包、函数
\end{verbatim}
\section{待完成插件}
\label{sec:org6baf975}
evil-leader
\begin{itemize}
\item 然后用其实现markdown的所有功能(非常轻松)
\item 可以用其方便的实现命令行功能(其实是emacs的功能),使用linux的指令。
\item 再之后可以尝试org-mode的gtd功能
\item 再之后可以尝试下org-mode的导出功能,导出html之类的不值得提,org-mode可以导出和导出思维导图。
\item 可以尝试一下org-mode的多文件查找,以及快速捕捉系统。
\item 再之后可以尝试下org-mode的对代码块的强大处理
\item 可以直接在代码块里执行c/c++/python等等语言,输出代码执行的结果,而无须切换回来
\item 可以尝试了解下org-mode的与其它组件的配合。以及可以了解下emacs和evil。另外也可以了解下配合git实现版本管理。
\item 最后,如果你对编程有些了解,那么所有你不满意的地方,都可以自己改,而不是向作者抱怨能否再下一版本实现某某功能。
\end{itemize}
\end{document}
